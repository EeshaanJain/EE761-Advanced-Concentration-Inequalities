\chapter{On Talagrand's Inequality}
\section{Talagrand's Inequality}
Theorem~\ref{thm:talagrand-inequality} showed that something similar to blowing-up lemma holds for the Talagrand convex distance. Moreover, we define:
\[
d_T(\ov x, A) = \sup_{\{\ov\alpha: \|\alpha\|_2 = 1\}} d_{\ov\alpha}(\ov x, A) =  \sup_{\{\ov\alpha: \|\alpha\|_2 = 1\}} \min_{\ov y \in A}d_{\ov\alpha} (\ov x, \ov y)
 \]
 \begin{note}
Even if we choose $\ov\alpha = \ov\alpha(\ov x)$, then $d_T(\ov x, \ov y) = d_{\ov\alpha(x)} (\ov x, \ov y)$
 \end{note}
 \section{Applications}
 \begin{definition}[Configuration function]\label{def:config-function}
$f:\RR^n \to \RR$ is a $c$-configuration function if $\forall x,y \in \Omega$,
\begin{equation}
    f(\ov x) \leq f(\ov y) + \sqrt{c f(\ov x)} d_{\ov \alpha} (\ov x, \ov y)
\end{equation}
for some $\ov \alpha (\ov x)$ (non-negative, unit norm).
\end{definition} 
\begin{remark}
Later: for configuration functions, Talagrand's inequality $\implies$ median concentrations. 
\end{remark}
Some examples of configuration functions are:
\subsection{Length of the longest increasing subsequence} 
Let $\mathcal{K}(\ov x) = \{l \subseteq \{1, \dots, n\}$. Let $\ell = |k|, k = \{i_1, \dots, i_\ell\}$ such that $i_1 < i_2 < \dots <i_\ell$, and $x_{i_1} < x_{i_2} < \dots < x_{i_\ell}\}$. $\mathcal{K}$ contains all increasing subsequences of $\ov x$. Then, $\text{inc}(\ov x) = \sup_{k \in \mathcal{K} (\ov x)} |k|$. 
\begin{prop}
    $\text{inc}(\ov x)$ is a 1-configuration function.
\end{prop}
\begin{proof}
$\forall x \in \Omega$, $\exists k(\ov x) \subseteq \{1, \dots, n\}$ s.t. inc$(\ov x)= |k|$.
Note that for $\ov y \in \Omega$, we can write $\text{inc}(\ov y) \geq |\{i \in k(\ov x) : y_i = x_i\}|$. This is obvious, as we are comparing only the same elements.
\begin{align*}
    \text{inc}(\ov y) &\geq |\{i \in k(\ov x) : y_i = x_i\}| \\
    &= \sum_{i \in k(\ov x)} \bm{1}_{\{x_i = y_i\}} \\
    &= |k(\ov x)| - \sum_{i \in k(\ov x)} \bm{1}_{x_i \neq y_i}
\end{align*}
Define $\alpha_i(\ov x) = \frac{\bm{1}_{i \in k(\ov x)}}{\sqrt{|k(\ov x)|}}$ for $i = 1, \dots, n$. We can write
Then we can write
\[
\text{inc} (\ov y) \geq \text{inc}(\ov x) - \sqrt{\text{inc}(\ov x)} d_{\ov \alpha}(\ov x, \ov y)
\]
\end{proof}
\subsection{Maximum length common subsequence}
\newcommand{\com}{\text{com}}
Let $\ov x = (x_1, \dots, x_{n_1})$, and $\ov y = (y_1, \dots, y_{n_2})$. Let $\ov z = (\ov y : \ov y)$ which has $n_1 + n_2 $ elements. Define $\com(\ov x, \ov y) =$ maximum length of any common subsequence between $\ov x$ and $\ov y$. An example is:
\[
\begin{split}
    \ov x &= \bm3, 4, \bm7, \bm9, 1, 3 \\
    \ov y &= 7, \bm3, \bm7, 1, \bm9, 4, 5
\end{split}
\implies \com(\ov x, \ov y) = 3
\]
Note that we could've also used $1$'s instead of $9$'s.
\begin{prop}
$\com(\ov x, \ov y)$ is a 2-configuration function.
\end{prop}
\begin{proof}
Let $\ov z = \ov x:\ov y$, where $\ov x, \ov a$ have $n_1$ elements, and $\ov w = \ov a: \ov b$ where $\ov y, \ov b$ have $n_2$ elements. Let $k_1 \subseteq \{1, 2, \dots, n_1\}$, and $k_2 \subseteq \{1, 2, \dots, n_2\}$ such that $|k_1| = |k_2| = \ell$. Further, $k_1 = \{i_r, r = 1, 2, \dots, \ell\}$ and $k_2 = \{j_r, r = 1, 2, \dots, \ell\}$, and $x_{i_r} = y_{j_r} \forall r \in \{1,2,\dots, \ell\}$.
\begin{align*}
\com(\ov a, \ov b) &\geq \left|\{r: 1\leq r\leq \ell, a_{i_r} = x_{i_r}, b_{j_r} = y_{j_r} \right| \\
&= \ell - \left|r: a_{i_r} \neq x_{i_r} \text{ or } b_{j_r} \neq y_{j_r}\right| \\
&= \com(\ov x, \ov y) - \sum_{r=1}^\ell \bm{1}_{(a_{i_r} \neq x_{i_r}) \cup (b_{j_r} \neq y_{j_r})} \\
&\geq \com(\ov x, \ov y) - \sum_{r=1}^\ell \bm1_{a_{i_r} \neq x_{i_r}} - \sum_{r=1}^\ell \bm{1}_{b_{j_r} \neq y_{j_r}}
\end{align*}
Let $\alpha_s = \bm 1_{\{s \in k_1 \text{ or } s \in n_1 + k_2\}}/{\sqrt 2\ell}$. The 2nd term in notation essentially comes as we appended $\ov y$ to $\ov x$ for the input. We can substitute the above to get the result.
\end{proof}
\section{Implication of $c-$configuration functions}
\begin{theorem}\label{thm:config-median}
Let $f(\ov X)$ be a $c$-configuration function, and $\ov X = (X_1, \dots, X_n)$ be independent random variables. Let $m = \text{median}(f(\ov X))$. Then, $\forall t \geq 0$:
\begin{equation}
\begin{split}
    \PP(f(\ov X )\geq m + t &\leq 2 \exp\left(-\frac{t^2}{4c(m+t)}\right) \\
    \PP(f(\ov X) \leq m - t &\leq 2 \exp\left(-\frac{t^2}{4cm}\right)
\end{split}
\end{equation}
\end{theorem}
\begin{proof}
To be done later.
\end{proof}
\begin{theorem}
Suppose $\forall \ov x$, if $\exists \alpha (\ov x)$ such that
\[
f(\ov x) \leq f(\ov y) + c d_{\ov \alpha}(\ov x, \ov y) \; \forall \ov y \in \Omega
\]
then,
\begin{equation}
    \PP(|f(\ov X) - m| \geq t) \leq 4\exp\left(-\frac{t^2}{4c^2}\right)
\end{equation}
(the setting is same i.e., $\ov X = (X_1, \dots, X_n)$ are independent).
\end{theorem}
\begin{proof}
To be done later.
\end{proof}
